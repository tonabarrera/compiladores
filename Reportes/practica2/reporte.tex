% !TeX spellcheck = es_ES
\documentclass[titlepage]{article}
\usepackage[utf8]{inputenc}
\usepackage[letterpaper, margin=2.5cm]{geometry}
\usepackage[spanish]{babel}
\usepackage{listings}
\usepackage{url}
\usepackage{float}
\usepackage{graphicx} 
\usepackage{color}

\usepackage[nottoc,notlot,notlof]{tocbibind}
\definecolor{dkgreen}{rgb}{0,0.6,0}
\definecolor{gray}{rgb}{0.5,0.5,0.5}
\definecolor{mauve}{RGB}{253,151,31}
\definecolor{deepred}{RGB}{249,38,114}

\lstset{frame=tb,
  language=Python,
  aboveskip=3mm,
  belowskip=3mm,
  showstringspaces=false,
  columns=flexible,
  numbers=left,
  stepnumber=1,
  basicstyle={\small\ttfamily},
  numberstyle=\tiny\color{gray},
  keywordstyle=\color{blue},
  commentstyle=\color{dkgreen},
  stringstyle=\color{mauve},
  breaklines=true,
  breakatwhitespace=true,
  tabsize=2,
  morekeywords={self, append},
  emph={Transicion, __init__, True, False, __str__, AFN, AFD, Analizador},
  emphstyle=\color{deepred}
}

\title{Reporte: Práctica 2}
\author{Barrera Pérez Carlos Tonatihu \\ Profesor: Saucedo Delgado Rafael Norman \\ Compiladores \\ Grupo: 3CM6 }
\begin{document}
  \maketitle
  \tableofcontents
  \newpage
  \section{Introducción}
  Existen diferentes formas para poder construir un autómata finito no determinista a partir de una expresión regular, en esta práctica 
  se utilizo la construcción de Thomson \cite{compis} con este algoritmo cada vez que se representa algún símbolo, una unión, concatenación, 
  cerradura de Kleene o positiva se crean dos nuevos estados. El algoritmo consiste en lo siguiente:
  \begin{itemize}
   \item Para $ \epsilon $ se construyen dos estados y la transición entre ellos se etiqueta con épsilon como en la figura \ref{fig:e}
      \begin{figure}[H]
	\begin{center}
	  \includegraphics[width=4cm]{e.png}
	  \caption{Épsilon representado en un autómata}
	  \label{fig:e}
	\end{center}
      \end{figure}
    \item Para algún símbolo del alfabeto se realiza lo mismo que para $ \epsilon $ pero esta vez se etiqueta con el respectivo símbolo como el ejemplo 
    de la figura \ref{fig:a}.
      \begin{figure}[H]
	\begin{center}
	  \includegraphics[width=4cm]{a.png}
	  \caption{Representación de la transición de un estado a otro a través del símbolo $a$.}
	  \label{fig:a}
	\end{center}
      \end{figure}
    \item La unión de dos expresiones regulares $ s $ y $ t $ se representa a través del uso de transiciones épsilon como se muestra en la figura 
    \ref{fig:union}.
      \begin{figure}[H]
	\begin{center}
	  \includegraphics[width=6cm]{union.png}
	  \caption{Unión de dos expresiones regulares ($s|t$) representadas en un AFN-$\epsilon$.}
	  \label{fig:union}
	\end{center}
      \end{figure}
    \item La concatenación de dos expresiones regulares $ s $ y $ t $ se realiza tomando el estado inicial de la transición $ t $ y convirtiéndolo 
    en el estado final de la transición producida por la expresión regular $ s $ como en la figura \ref{fig:concatenacion}.
      \begin{figure}[H]
	\begin{center}
	  \includegraphics[width=6cm]{concatenacion.png}
	  \caption{Concatenación ($st$) mediante la fusión de un estado final y uno inicial respectivamente.}
	  \label{fig:concatenacion}
	\end{center}
      \end{figure}
    \item La cerradura de Kleene de una expresión regular $ s^* $ se realiza usando transiciones épsilon entre los estados que componen esta expresión 
    Esto se muestra en la figura \ref{fig:cerradura}.
      \begin{figure}[H]
	\begin{center}
	  \includegraphics[width=6cm]{cerradura.png}
	  \caption{Cerradura ($s^*$).}
	  \label{fig:cerradura}
	\end{center}
      \end{figure}
    \item La cerradura positiva es similar a la de Kleene pero esta no realiza una transición del estado inicial al final.
  \end{itemize}
  Finalmente, para la evaluación de la expresión regular primero la pasamos a postfijo con el algoritmo shunting yard \cite{postfijo} que se utiliza para la 
  evaluación de expresiones matemáticas pero adaptándolo a expresiones regulares.
  \section{Desarrollo}
  \begin{lstlisting}
from automata.automatas import Transicion
from automata.automatas import AFN

class Analizador:
	def __init__(self):
		self.salida_postfijo = list()
		self.transiciones = list()
		self.pila_transiciones = []
		self.POSITIVA = 0
		self.KLEENE = 1
		self.AFN = AFN()
	
	def mostar_expresion_postfijo(self):
		print(self.salida_postfijo)
	
	def mostar_automata(self):
		print('Inicial: %s Finales: %s' % (self.AFN.estado_inicial, self.AFN.estados_finales))
		print("Transiciones:")
		for t in self.AFN.transiciones:
		print(t)
	
	def convertir_postfijo(self, cadena):
		pila = []
		punto = False
		for c in cadena:
			if c == '(':
				if punto:
					while len(pila) > 0 and (pila[-1] == '+' or pila[-1] == '*'):
						self.salida_postfijo.append(pila.pop())
					pila.append('.')
					punto = False
				pila.append(c)
			elif c == ')':
				punto = True
				while len(pila) > 0 and pila[-1] != '(':
					self.salida_postfijo.append(pila.pop())
				try:
					pila.pop()
				except IndexError as e:
					raise e
			elif c == '+' or c == '*':
				self.salida_postfijo.append(c)
			elif c == '|':
				while len(pila) > 0 and (pila[-1] == '+' or pila[-1] == '*' or pila[-1] == '.'):
					self.salida_postfijo.append(pila.pop())
				pila.append(c)
				punto = False
			else:
				if punto:
					while len(pila) > 0 and (pila[-1] == '+' or pila[-1] == '*'):
						self.salida_postfijo.append(pila.pop())
				pila.append('.')
				punto = True
				self.salida_postfijo.append(c)
		
		while len(pila) > 0:
			self.salida_postfijo.append(pila.pop())
	
	def generar_automata(self):
		inicial = 1
		final = 2
		self.AFN.alfabeto.add('e')
		for c in self.salida_postfijo:
			if c == '*':
				print('CERRADURA KLEENE')
				s = self.pila_transiciones.pop()
				self.cerradura(s, self.KLEENE)
			elif c == '+':
				print('CERRADURA DE POSITIVA')
				s = self.pila_transiciones.pop()
				self.cerradura(s, self.POSITIVA)
			elif c == '|':
				print("UNION")
				s = self.pila_transiciones.pop()
				t = self.pila_transiciones.pop()
				i1 = Transicion(s[1] + 1, s[0], 'e')
				i2 = Transicion(s[1] + 1, t[0], 'e')
				f1 = Transicion(s[1], s[1] + 2, 'e')
				f2 = Transicion(t[1], s[1] + 2, 'e')
				self.pila_transiciones.append([s[1] + 1, s[1] + 2])
				self.transiciones.extend((i1, i2, f1, f2))
			elif c == '.':
				print('CONCATENACION')
				t = self.pila_transiciones.pop()
				s = self.pila_transiciones.pop()
				for aux in self.transiciones:
					if aux.siguiente == s[1]:
						aux.siguiente = t[0]
				self.pila_transiciones.append([s[0], t[1]])
			else:
				print('ESTADO')
				if self.pila_transiciones.__len__() > 0:
					inicial = self.pila_transiciones[-1][1] + 1
					final = inicial + 1
				transicion = Transicion(inicial, final, c)
				self.pila_transiciones.append([inicial, final])
				self.transiciones.append(transicion)
				self.AFN.alfabeto.add(c)
		
		self.AFN.agregar_inicial(self.pila_transiciones[0][0])
		self.AFN.agregar_finales({self.pila_transiciones[0][1]})
		self.AFN.transiciones = self.transiciones
	
	def cerradura(self, s, tipo):
		i1 = Transicion(s[1] + 1, s[0], 'e')
		s1 = Transicion(s[1], s[0], 'e')
		s2 = Transicion(s[1], s[1] + 2, 'e')
		self.pila_transiciones.append([s[1] + 1, s[1] + 2])
		self.transiciones.extend((i1, s1, s2))
	
		if tipo == self.KLEENE:
			i2 = Transicion(s[1] + 1, s[1] + 2, 'e')
			self.transiciones.append(i2)
  \end{lstlisting}
  \section{Resultados}
  A continuación se presenta el resultado de la implementación de la clase \emph{Analizador} en donde se convierte una expresión regular 
  a un AFN para después evaluar algunas cadenas y ver si son validas. Las pruebas se realizaron con el siguiente código, el cual ocupa la clase 
  \emph{AFN} que se construye gracias a la clase \emph{Analizador} para poder llevar a cabo la evaluación de algunas cadenas.
  
  \begin{lstlisting}
  def correr_analizador():
	  print("Generando el automata de la expresion: (a|b)*abb ...")
	  analizador = Analizador()
	  # Transformamos la expresion a postfijo
	  analizador.convertir_postfijo('(a|b)*abb')
	  analizador.mostar_expresion_postfijo()
	  # Creamos el AFN a partir de la expresion en postfijo
	  analizador.generar_automata()
	  analizador.mostar_automata()
	  # Probamos nuestro automata con algunas cadenas
	  print("Pruebas sobre el automata generado...")
	  automataAFN = analizador.AFN
	  for n in range(5):
		  cadena = input("-Ingresa una cadena: ")
		  print("La cadena es:")
		  if automataAFN.evaluar_cadena(cadena):
		  	print("Valida")
		  else:
		  	print("No valida")
  \end{lstlisting}
  
  \begin{figure}[H]
    \begin{center}
      \includegraphics[width=10cm]{generacion.png}
	\caption{Transformando la expresión regular $ (a|b)^{*}abb $ a su respectivo autómata}
	\label{fig:generacion}
    \end{center}
  \end{figure}
  
  Como se puede observar en la figura \ref{fig:generacion} la expresión regular se pasa a notación posfija y después se puede observar como se realiza 
  la creación de las transiciones, lo siguiente que aparece es mostrar los estados inicial y final y finalmente se muestran las transiciones resultantes. 
  
  Después se realizaron pruebas sobre el autómata que se genero de forma similar a la práctica anterior.
  
  \begin{figure}[H]
    \begin{center}
      \includegraphics[width=10cm]{pruebas.png}
	\caption{Pruebas sobre el autómata generado.}
	\label{fig:pruebas}
    \end{center}
  \end{figure}
  
  La generación del autómata se realizo de forma correcta (figura \ref{fig:pruebas}), ya que la evaluación de las cadenas fue satisfactoria esto quiere decir que las transiciones 
  que se muestran en la figura \ref{fig:generacion} no tuvieron errores.
  
  \section{Conclusiones}
  La elaboración de esta práctica fue un poco más sencilla que la anterior debido a que solo consistió en implementar el algoritmo de Thomson 
  y con ello poder construir cualquier AFN, la parte complicada de modelar este algoritmo fue la concatenación ya que requirió unos pasos extra 
  a diferencia del resto de construcciones. 
  
  Y como se pudo observar en las pruebas la conversión de expresión regular a autómata se realiza correctamente. 
  Esto demuestra la versatilidad que tienen las expresiones regulares y los autómatas además de que sera de 
  mucha utilidad cuando realicemos nuestro analizador léxico que su principal tarea es revisar el vocabulario de nuestro compilador.
  \bibliography{bibliografia} 
  \bibliographystyle{ieeetr}
\end{document}