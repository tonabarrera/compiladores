% !TeX spellcheck = es_ES
\documentclass[12pt, titlepage]{article}
\usepackage[letterpaper, margin=2.5cm]{geometry}
\usepackage[utf8]{inputenc}
\usepackage[spanish]{babel}

\usepackage{float}
\usepackage{graphicx}

\usepackage{color}
\usepackage{listings}
\usepackage[nottoc,notlot,notlof]{tocbibind}
\definecolor{dkgreen}{rgb}{0,0.6,0}
\definecolor{gray}{rgb}{0.5,0.5,0.5}
\definecolor{mauve}{RGB}{253,151,31}
\definecolor{deepred}{RGB}{249,38,114}
\lstset{frame=tb,
language=Python,
aboveskip=3mm,
belowskip=3mm,
showstringspaces=false,
columns=flexible,
numbers=left,
stepnumber=1,
basicstyle={\small\ttfamily},
numberstyle=\tiny\color{gray},
keywordstyle=\color{blue},
commentstyle=\color{dkgreen},
stringstyle=\color{mauve},
breaklines=true,
breakatwhitespace=true,
tabsize=2,
morekeywords={self, append},
emph={Transicion, __init__, True, False, __str__, AFN, AFD, Analizador},
emphstyle=\color{deepred}
}

%opening
\title{Reporte: Práctica 5}
\author{Barrera Pérez Carlos Tonatihu \\ Profesor: Saucedo Delgado Rafael Norman 
\\ Compiladores \\ Grupo: 3CM6}

\begin{document}

\maketitle
\tableofcontents
\newpage
\section{Introducción}
\section{Desarrollo}
\section{Resultados}
\newpage
\section{Conclusiones}
El analizador sintáctico por descenso recursivo es bastante sencillo por lo que 
es fácil de entender el como funciona y para ejemplos como el trabajado en esta 
practica es útil debido a que la gramática es bastante simple, esto se ve 
reflejado en el hecho de que la parte más difícil de esta práctica fue definir 
una correcta gramática que permitiera modelar el lenguaje que se trabajo y el 
resto fue abstraer dicha gramática a nivel programación.
\\\\
Sin embargo, al trabajar con ejemplos más complejos este tipo de analizador 
sintáctico no sera tan conveniente debido a las limitaciones que presenta para 
su correcto funcionamiento.
A pesar de esto el conocer como funciona permite entender el funcionamiento del 
resto de analizadores sintácticos y el porque se su existencia.
\bibliography{bibliografia} 
\bibliographystyle{ieeetr}

\end{document}
