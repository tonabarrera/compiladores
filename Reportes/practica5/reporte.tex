% !TeX spellcheck = es_ES
\documentclass[12pt, titlepage]{article}
\usepackage[letterpaper, margin=2.5cm]{geometry}
\usepackage[utf8]{inputenc}
\usepackage[spanish]{babel}

\usepackage{float}
\usepackage{graphicx}

\usepackage{color}
\usepackage{listings}
\usepackage[nottoc,notlot,notlof]{tocbibind}
\definecolor{dkgreen}{rgb}{0,0.6,0}
\definecolor{gray}{rgb}{0.5,0.5,0.5}
\definecolor{mauve}{RGB}{253,151,31}
\definecolor{deepred}{RGB}{249,38,114}
\lstset{frame=tb,
language=Python,
aboveskip=3mm,
belowskip=3mm,
showstringspaces=false,
columns=flexible,
numbers=left,
stepnumber=1,
basicstyle={\small\ttfamily},
numberstyle=\tiny\color{gray},
keywordstyle=\color{blue},
commentstyle=\color{dkgreen},
stringstyle=\color{mauve},
breaklines=true,
breakatwhitespace=true,
tabsize=2,
morekeywords={self, append},
emph={Transicion, __init__, True, False, __str__, AFN, AFD, Analizador},
emphstyle=\color{deepred}
}

%opening
\title{Reporte: Práctica 5}
\author{Barrera Pérez Carlos Tonatihu \\ Profesor: Saucedo Delgado Rafael 
Norman 
\\ Compiladores \\ Grupo: 3CM6}

\begin{document}

\maketitle
\tableofcontents
\newpage
\section{Introducción}
Un analizador por descenso recursivo es un analizador sintáctico de arriba hacia abajo en el cual un conjunto de métodos recursivos son usados para procesar una entrada.
El análisis de arriba hacia abajo puede ser visto como un intento de encontrar la derivación mas a la izquierda para una cadena.
Esto genera que el árbol de sintaxis sea construido desde la raíz y creando nodos en preorden. \cite{compis}

En este tipo de analizador se tiene un método asociado a cada símbolo no terminal de la gramática
\section{Desarrollo}
\section{Resultados}
\newpage
\section{Conclusiones}
El analizador sintáctico por descenso recursivo es bastante sencillo por lo que 
es fácil entender el como funciona es por esto que para ejemplos como el 
trabajado en esta 
practica es bastante útil debido a que la gramática también es simple, esto se 
ve 
reflejado en el hecho de que la parte más difícil de esta práctica fue definir 
una correcta gramática que permitiera modelar el lenguaje que se trabajo y el 
resto fue abstraer dicha gramática a nivel programación.
\\\\
Sin embargo, al trabajar con ejemplos más complejos este tipo de analizador 
sintáctico no sera tan conveniente debido a las limitaciones que presenta para 
su correcto funcionamiento como lo es el hecho que una gramática que sea 
recursiva por la izquierda puede provocar que se entre en un bucle infinito 
debido a que si nos encontramos una producción que intente expandir un símbolo 
no terminal \emph{A} al tener dicha recursión nos volveremos a encontrar con la 
situación de expandir otra vez a \emph{A}.
A pesar de esto el conocer como funciona permite entender el funcionamiento del 
resto de analizadores sintácticos y el porque se su existencia.
\bibliography{reporte} 
\bibliographystyle{ieeetr}

\end{document}
